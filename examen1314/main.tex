\documentclass[a4paper]{article}

%% Language and font encodings
\usepackage[dutch]{babel}
\usepackage[utf8x]{inputenc}
\usepackage[T1]{fontenc}

%% Sets page size and margins
\usepackage[a4paper,top=3cm,bottom=3cm,left=3cm,right=3cm,marginparwidth=1.75cm]{geometry}

%% Useful packages
\usepackage{amsmath}

\title{Examen}
\author{Achim Vandierendonck}
\date{2013 - 2014}

\begin{document}
\maketitle

\section{Theorie}
\begin{enumerate}
	\item We hebben een functie $R(t)$ (waarvoor $R(0) > R(t), \forall t \neq 0$) en hebben dus reden om aan te nemen dat het gaat over een autocorrelatiefunctie.
	\begin{enumerate}
		\item Hoe kunnen we dit zeker zijn?
		\item Hoe zouden we er een echte autocorrelatiefunctie van kunnen maken. \textit{Denk aan Wiener filter.}
	\end{enumerate}
	\item Hoe kunnen we via de FFT snel de autocorrelatie berekenen? Enkel de procedure, geen formules omtrent de efficiëntie.
	\item Hoe kunnen we de periode van een signaal afleiden uit het amplitudeceptstrum (zie vergelijking \ref{eq:ceptstrum})?
\end{enumerate}

\begin{equation}
	|\mathcal{F}^{-1}\{\log{|\mathcal{F}\{f(t)\}|^2}\}|^2
	\label{eq:ceptstrum}
\end{equation}

\section{Oefeningen}
Gevraagd wordt om een IIR filter van 4e orde te ontwerpen waarbij de passband rond $fTs = 1/3$ ligt met een bandbreedte van $1/20\pi$. Het is mogelijk om dit te verwezenlijken door de som of het verschil te nemen van 2 tweede orde filters.
\begin{enumerate}
	\item Bespreek het principe omtrent hoe je dit zou kunnen verwezenlijken. \textit{Dit is analoog aan de oefeningenles omtrent subbandcodering, 2 filters laten snijden in het $3dB$ punt.}
	\item Waarom levert enkel het verschil een goed resultaat?
	\item Bereken de polen en nullen van dit filter
	\item Met welk van de 4 types filters kan je dit het best vergelijken?
\end{enumerate}

\end{document}